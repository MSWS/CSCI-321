\documentclass{article}

\usepackage{amsmath, amssymb}
\usepackage[a4paper, margin=1in]{geometry}

\usepackage{libertine}
\usepackage{libertinust1math}
\usepackage[T1]{fontenc}

\title{Lecture Exercise 3}
\author{Isaac Boaz}

\begin{document}
\maketitle

\begin{enumerate}
    \item What does `set dressing' refer to in the context of building a level? \\
          Set dressing refers to cosmetic elements that are added to a level to
          make it more visually appealing. This includes things like props,
          lighting, and other visual elements that do not affect gameplay.
    \item What is a static mesh and why label some meshes as static? \\
          A static mesh is a mesh that does not move or change during gameplay.
          This is useful primarily for performance optimizations, especially
          when it comes to rendering / lighting conditions.
          By marking a mesh as static, you are telling Unity that it does not
          need to do some computations that would be necessary if the mesh were
          dynamic.
    \item What is a heightmap used for?
          A heightmap is used to represent the elevation of a terrain. It is a
          2D array of values that represent the height of the terrain at each
          point. This is useful for generating realistic terrain, as well as
          for physics calculations.
    \item How do billboards, which are 2d visuals, give the effect of a 3D
          model? \\
          Billboards are 2D visuals that always face the camera. This
          gives the illusion of a 3D model because the billboard is always
          facing the camera, so it appears to be a 3D object. This is useful
          for things like trees, grass, and other objects that are far away
          and do not need to be rendered in full 3D.
    \item What is the advantage of whiteboxing environments during game
          development? \\
          Whiteboxing allows for easy and quick prototyping of a level or
          game. It is also a good first step for game development, as further
          enhancements and details (set dressing) can be iterated on top of the
          whitebox.
\end{enumerate}

\end{document}