\documentclass{article}

\usepackage{amsmath, amssymb}

\usepackage{titlesec}
\titleformat{\section}[block]{\Large\bfseries\filcenter}{}{1em}{\thesection\ - }

\usepackage{libertine}
\usepackage{libertinust1math}
\usepackage[T1]{fontenc}

\usepackage[a4paper]{geometry}
\usepackage{multicol}

\title{Game Journal}
\author{Isaac Boaz}

\newcommand\wordcount{
    \immediate\write18{texcount -sub=section \jobname.tex  | grep "Section" | sed -e 's/+.*//' | sed -n \thesection p > 'count.txt'}
(\input{count.txt}words)}

\renewcommand{\contentsname}{Table of Contents}
\makeatletter
\renewcommand\tableofcontents{%
  \begin{center}
  \null\textbf{\Large\contentsname}\null\par
  \end{center}
  \@mkboth{\MakeUppercase\contentsname}{\MakeUppercase\contentsname}%
  \@starttoc{toc}%
}
\makeatother

\begin{document}

\maketitle

\tableofcontents

\pagebreak

\section{10/10/2024 - Risk of Rain 2}
\begin{center}
      \textit{Platform: PC} | \textit{Word Count: 717}
\end{center}
Risk of Rain 2 is a third-person roguelike game that tasks the player with
fighting against increasingly difficult waves of enemies. While I have not
beaten the game (yet), the game has a fairly straightforward gameplay loop.
The game is split over multiple levels (\textasciitilde 5) in different
settings. Each level has a teleporter that the player must find in order to
progress to the next stage.

Stages seem to be somewhat procedurally generated, though each stage in terms
of progression has a similar `theme'. Once a player is in a world, monsters
immediately start spawning randomly within a certain radius of the player. The
core difficulty lies here: the player must balance killing monsters with
finding the teleporter.

Around each level there are chests or other power-ups (e.g. shrines) that the
player must spend money to open. The player gains this money by killing
monsters, adding an additional incentive (besides safety) to killing the
monsters. This mechanic also introduces the idea of not immediately taking the teleporter
before you have enough power-ups to survive the next level.

Astute readers will notice that this seems similar to many RPGs, where it is
possible to over-level oneself to easily defeat the next level / boss. Risk of
Rain 2 disencentivizes this by having a (very visible) difficulty scale that
slowly increases the longer you are in a level. The longer you are in a level,
the more gold you can obtain. Risk of Rain 2 disencentivizes this by having a (very visible) difficulty scale that
slowly increases the longer you are in a level.

\subsection{Good}
\begin{description}
      \item[Multiple Characters] Risk of Rain 2 starts you out with 3 basic
            characters, with at least 12 more to unlock. A major benefit to this is
            that it allows the player to experiment with different playstyles. Some
            players may prefer to have a gun where they can precisely point and click, whereas others may wish
            to have a bow and arrow that auto-targets enemies for them.
      \item[UI] The UI is clean and unobtrusive, while still providing necessary
            information (difficulty, weapons, upgrades, etc.) Depending on the
            character used, the UI will change to reflect the character's abilities.
      \item[UX] The performance and graphics of the game allow for smooth gameplay
            on reasonably powereful hardware. The game offers both a low skill
            floor, and a high skill ceiling. Players are able to learn the attack
            patterns of common enemies, and the teleporter has additional particle
            effects that attentive players can use to locate it.
      \item[Sound] Nearly every action taken within the game has an associated
            and distinct sound to it. Different weapons sound different, enemies
            emit sounds occasionally, and dealing damage has a clear indication.
      \item[Multiplayer] The game supports up to 4 player multiplayer,
            including support for `pinging' locations (e.g. teleporters, items,
            upgrades, etc.)
\end{description}
\subsection{Bad}

\begin{description}
      \item[Weapon Balancing] Oftentimes characters have a `primary' attack that
            players find themselves depending on for the majority of the game. One
            example is a character with a shotgun that shoots bulletes in a horizontal
            pattern. The character's secondary attack is a knife that they use to slash
            (for close range) attacks. I have found that I very rarely use the secondary
            attack, and the game does not seem to reward diversifying the attacks or
            weapons that are used.
      \item[Progression] Despite playing the game for 8+ hours, I have
            `only' unlocked one additional character. Without wanting to spoil the
            game, I am unsure of what potential unlockables, (permanent) upgrades,
            or other progression mechanics the game has. This makes the game feel
            slightly shallow at the moment, but this is partially a \emph{skill issue}.
      \item[Difficulty] With so many characters, it does seem that some are more
            catered to specific playstyles. In addition to `recommended' controls, it
            would be nice to have an indication of how `easy' a character is to play,
            including what controller would be recommended.
            However, allowing players to experiment with the characters and have them
            find out for themselves is also a valid design choice.
\end{description}

\subsection{Summary}
Overall, Risk of Rain 2 is a fun third-person shooter for those who have
\textasciitilde 30 minutes to an hour to kill. With a low skill floor and high
skill ceiling, the game is easy to pick up and play, but difficult to master.

Replayability is high thanks to the vast characters, upgrades, and randomized
level generation.

\pagebreak

\section{10/17/2024 - Oxygen Not Included}

\begin{center}
      \textit{Platform: PC} | \textit{Word Count: 609}
\end{center}

Oxygen Not Included (ONI) is a 2D base-building survival game that allows the
player to manage a group of colonists that crash landed on an asteroid.
Management is heavily focused on resource management, including (unsurprisingly)
oxygen. This game is similar to Factorio in terms of logistics; all resources
have to be collected somehow, and the game leans heavily into automation and
managing the colonists.

Progression is up to the player, but the end goal is to launch a rocket (similar
to Factorio). Since the colonists are on an asteroid, there are natural
incentives to properly managing your resources. From waste management,
breathable oxygen, and even avoiding heat death, ONI is an in-depth game that
has a lot of depth.

\subsection{Good}
\begin{description}
      \item[Introduction] Despite the name of the game, the very beginning of
            the game \emph{does} include some oxygen. Some helpful hints and popups
            are admittedly a lazy way to introduce the player, but they are
            effective. The game manages a decent balance between lettings
            players figure out what does what, while at the same time
            introducing and explaining key mechanics.
      \item[UX] As the game requires good resource management, the game's UX
            allows the player to `pin' certain resources to the UI to quickly glance
            at what resources are most important to them. Cyclic summaries and graphs
            are also reported, allowing the player to see how their colony is doing.
      \item[Difficulty] The game has an extremely steep learning curve, but
            allows players to learn and understand machines within the game itself
            through an in-game `encyclopedia'. The game also has a moderately
            configurable difficulty setting, determining how quickly colonists
            have to eat, how easily they get stressed, and many other factors.
      \item[Modability] ONI supports the Steam Workshop, which allows
            programmers to write custom mods and addons for the game. This allows
            players that are looking for an extra challenge (or an easier time) to
            change how they want to play the game.
      \item[Controls] ONI keeps the controls simple with a good UX. A `priority'
            system allows the player to easily specify granular importance over what
            tasks should be done first. Clicking on a specific colonist also
            shows a window with all the information about that colonist,
            including what the current task is, and what they plan on doing next.
\end{description}

\subsection{Bad}
\begin{description}
      \item[Learning Curve] Though the game does its best to teach the player
            what is what, it is still very complex. It is not uncommon for players to
            rely on external resources (usually YouTube videos) as a guide for how to
            get started. Even intermediate players can find a specific task confusing
            or unclear how to accomplish.
      \item[Progression] Similar to the above issue, the game does not have a
            clear progression path. While the end goal is to launch a rocket, the
            game does not provide a clear path to get there. This can be frustrating
            for players who are looking for a clear goal to work towards. I have
            found myself restarting the game multiple times because it was
            unclear what the next steps are for my colony.
      \item[Default Settings] Thanks to the game's high replayability, players
            often will create a new game and immediately have preferences that they
            want to enforce. It is a bit of a chore to have to change global settings
            for the colony every time a new game is started (e.g. colony schedules).
\end{description}

\subsection{Summary}
ONI is a fun and challenging 2D base-building survival game that allows the
player to go as quick or as slow as they are comfortable with. The game has many
elements to its gameplay, and in conjunction with the game's randomization and
varying biomes, the game has a high replayability factor.

\pagebreak

\section{10/27/2024 - Sea of Thieves}

\begin{center}
      \textit{Platform: PC} | \textit{Word Count: 722}
\end{center}

Sea of Thieves is a first-person pirate game that tasks the player with running
a ship, usually with a crew. The game is similar to a rouge-like in that the
basic progression starts out at a common hub, with a few different ships
depending on the number of players. Crews are able to choose a quest to go on,
or simply sail around and explore the world.

Throughout the adventures, players can run into random events such as krakens,
megalodons, or other players. The game has a heavy emphasis on teamwork, where
one player could be steering the ship, another navigating and giving directions,
another repairing the ship, etc. In contrast to the actual ocean, the islands in
Sea of Thieves are relatively small and clustered together.

In combination with the game's great graphics and sound, Sea of Thieves is a
great game for players who enjoy teamwork and exploration.

\subsection{Good}
\begin{description}
      \item[Multiplayer] A core element to Sea of Thieves is the multiplayer
            support. Cross-platform between the varying consoles and PC allows players
            to player with their friends, regardless of what platform they are
            on. Another great feateure is the separation of `High-Seas' and
            private servers, allowing players to choose how they want to play.
            High-Seas servers are public and allow for random encounters with
            other players, whereas private servers only have the players that
            are part of the crew. The game manages a balance in this regard by
            having some quests that are only available on High-Seas servers.
      \item[Communication] The game has multiple ways of communicating with your
            crew and others, including a voice chat system, text chat, and
            emotes. Additional diagetic methods such as what pirate flag you are
            flying can also be used to communicate with other players what your
            preferred playstyle is (e.g. white flag for peace, black flag for
            war).
      \item[Quests] A wide selection of quests allows crews to tailor which
            quests they go on. Whether a crew prefers a heavier combat focus, puzzle
            solving, or exploration, there are quests for everyone. A
            combination of these quests as well as the random encounters tasks
            everyone to do a little bit of everything.
      \item[Graphics] Water is infamously difficult to render in games, but Sea
            of Thieves' comical approach allows for players to not be distracted, and
            is still very appreciable. This combined with the consistent art style and
            solid character designs makes the game visually appealing.
      \item[UX] Where possible the game conveys information diagetically. The
            game's strong commitment to the pirate / sailing theme is evident in the
            game's UI. One example is the mast, which visually blocks whoever is
            steering the ship. The game forces players to consider what sailing
            was like, including how crews communicated, navigated, and repaired ships.
\end{description}

\subsection{Bad}
\begin{description}
      \item[Progression] Due to the game being a rogue-like, progression has for
            the most part been cosmetic. Quests offer gold which can be used to buy
            cosmetic additions to ships, such as fancy cannon designs or sails.
            Though I have not played the game for super long, I haven't found
            another usage for gold yet. As long as you have friends to play with
            I don't see this as a major issue, but for solo players, there does
            seem to be less incentive to play.
      \item[Combat] Combat in Sea of Thieves is fairly straight forward. Players
            are given some cannonballs for their ships, a gun, and a sword.
            Combat on the ship is more invovled, allowing players to use a
            harpoon or an array of cannonballs to attack other ships. However,
            once off the ship, combat is fairly simple.
      \item[Exploration] Though an important aspect of the game, the exploration
            and new discoveries the game offers seems to be lacking. While a
            consistent art style is nice, the game's islands are fairly similar to
            each other. This combined with most of the islands being within a 10
            minute sailing distance makes the game feel a bit shallow.
\end{description}

\subsection{Summary}
Sea of Thieves is a fun and engaging pirate game that is best played with a
group of friends. The emphasis on teamwork and communication allows for involved
roleplaying with a nice skill ceiling if playing on a High-Seas server. Quests
that can be tailored to the crew's playstyle, and the game's great graphics and
soundtrack make Sea of Thieves a great game for those who enjoy exploration and
teamwork.

\pagebreak

\section{11/03/2024 - Peggle Deluxe}

\begin{center}
      \textit{Platform: PC} | \textit{Word Count: 571}
\end{center}

Peggle (Deluxe) is a casual pachinko-like puzzle game that tasks the player with
clearing a field of `pegs' by shooting a ball from the top which will fall down
to the bottom. Each peg has a varying color determining its function, with
orange pegs being the level objective, blue pegs being obstacles (and worth
small points), purple pegs being worth more points, and green pegs granting a
power depending on which character you are using.

Peggle Deluxe has 10 characters, each with their own unique power. Alongside
each character are 5 levels tailored to that character's power. Adding on 5
bonus levels where you can use any character, Peggle Deluxe offers a total of 55
levels. Character powers vary from an extended guide, a multi-ball, an auto-aim,
and even lobster claws on the side!

\subsection{Good}
\begin{description}
      \item[Simple] Peggle Deluxe is a simple game that is easy to understand. A
            skill floor combined with a high skill ceiling allows new players to get
            through the game reasonably well, and experienced players to focus more on
            getting a high score, or speedrunning the game.
      \item[Randomization] Though the physical location of all level's pegs are
            set, the color (and thus behavior) of each peg is randomized. This both
            allows for `opening strategies' to be developed for the first shot, but
            also requires experienced players to adapt to the randomization.
      \item[Characters] With 10 total characters, each power allows players to
            have  preferred playstyle and strategy. Once a player completes all levels
            through the story mode with their pre-set characters, they can then go
            back and play through all levels with any character they pick.
      \item[Challenge] Though the game itself can be beaten within a few hours,
            the replayability comes from trying to get a high score or imposing other
            limitations to yourself (eg: no powerups). The game also has a few other
            challenge modes that task the player with reaching a certain score,
            increasing the \# of orange pegs, or reducing the number of balls
            you get.
      \item[Juice] The game has some nice sounds, camera and particle effects to
            make the game feel more responsive. The peg hit sound as you hit
            consecutive pegs with a single shot goes up in pitch, and on the last
            orange peg the camera zooms in as your ball goes near it.
\end{description}

\subsection{Bad}
\begin{description}
      \item[Luck] Though the game certainly has a skill aspect to it, there are
            some times where the game simply generates a level that requires a lot of
            luck to beat. This can be frustrating when a level goes from extremely
            hard to extremely easy once it's restarted.
      \item[Controls] A very small issue of the game is on the main menu screen.
            Though the game does allow for the shooter to be moved around both with
            the mouse and keyboard, the game's main menu can only be opened by
            clicking on the menu button, and not by pressing the escape key.
      \item[Replayability] Though the game has a decent amount of replayability,
            I felt that the game could have had more levels. The game's 55 levels can
            be beaten in a few hours, and the game does not have a level editor or
            other user-generated content.
\end{description}

\subsection{Summary}
Peggle Deluxe is a fun casual game that has decent replayability and allows
players to finetune their skill if they desire. The game's simple mechanics and
high skill ceiling make it a great game for those who enjoy skill-based puzzle
games. I was originally introduced to Peggle Deluxe through a speedrunner who
showcased the game's high skill ceiling alongside fun and engaging gameplay.

\pagebreak

\section{11/17/2024 - Stardew Valley}

\begin{center}
      \textit{Platform: PC} | \textit{Word Count: 583}
\end{center}

Stardew Valley is a casual 2D farming simulation game that introduces itself
with a custcene where the player becomes tired of their corporate life and comes
to the realization that their grandfather left them a farm in their inheritance.
The player moves to the farm, and the game begins.

Stardew Valley has an amazing amount of depth to it; the player can grow crops,
raise animals and friendships, go mining, fishing, and foraging. Tool upgrades
as well as a solid combat system with many weapons means that there is a lot for
everyone.The game's day and night cycle grants a decent amount of leeway in
terms of what they want to do per day, while at the same time requiring the
player to plan out what they want to do.

Developed by ConcernedApe, Stardew Valley is one of the best games period, and
the fact that a single person created this game makes it even more impressive.

\subsection{Good}
\begin{description}
      \item[Save System] A feature that can go completely unappreciated is the
            game's save system. The game saves at the end of every day, either when
            the player goes to bed or they pass out (at 02:00). This mechanic
            has a great balance in terms of both pros and cons and convenience.
            On one hand, the player can't do small tasks and then save the
            game, but on the other, if the player wants to redo a day entirely,
            they are easily able to do so.
            This type of system also sends a message to the player that you
            don't have to do everything possible in a day, as you will want to
            save eventually.
      \item[Depth] Stardew Valley has a lot of depth to it. With five skills,
            four types of weapons, half a dozen locations, a plethora of crops and
            animals, dozens of characters with many being candidates for marriage, the
            game seems to truly never end.
      \item[Information] Despite being a game with a lot of depth, the basic
            mechanics and controls of the game are fairly intuitive and easy to
            pickup. The game starts you off with a limited backpack size, and
            you can purchase more space as you progress. The game slowly
            introduces more mechanics as you progress; for example, initially
            the caves (where you go mining) are blocked off, you gain a fishing
            rod on the second day, you learn crafting recipes as you level up,
            etc. This slow introduction of mechanics allows the player to learn
            the ropes as they go, while also allowing experiences players to
            quickly access the mechanics they want.
      \item[Events] The game has a lot of events that occur throughout the
            calendar, including fairs, art shows, heart events, holidays, etc.
            As the player progresses, they will also unlock additional events
            based on some criteria (e.g. friendship level, season, etc.)
\end{description}

\subsection{Bad}
\begin{description}
      \item[Fishing] Perhaps the only weak part of Stardew Valley is the fishing
            minigame. The fishing minigame is one of the most divisive parts of
            the game, which goes to show how well the game is made. Some players
            consider the fishing minigame to be too boring, too difficult, or
            too easy. It is admittedly a bit of a chore, but I personally think
            it has a good balance of difficulty and reward.
\end{description}

Stardew Valley is an absolute must play for any casual gamer or someone who is
just getting into video games. The game has an insame amount of depth,
replayabiliy, progression, and is a testament to what a videogame can be.

\end{document}